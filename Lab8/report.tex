\documentclass[12pt,a4paper]{article}
\usepackage[utf8]{inputenc}
\usepackage{amsmath}
\usepackage{amsfonts}
\usepackage{amssymb}
\usepackage{graphicx}
\usepackage{listings}
\usepackage{xcolor}
\usepackage{geometry}
\usepackage{fancyhdr}
\usepackage{titlesec}

\geometry{margin=1in}
\pagestyle{fancy}
\fancyhf{}
\rhead{\thepage}
\lhead{Assembly Language Lab Report}

% Code listing style
\lstdefinestyle{assembly}{
    language=[ARM]Assembler,
    basicstyle=\ttfamily\footnotesize,
    keywordstyle=\color{blue}\bfseries,
    commentstyle=\color{green!60!black},
    stringstyle=\color{red},
    numbers=left,
    numberstyle=\tiny\color{gray},
    stepnumber=1,
    numbersep=5pt,
    backgroundcolor=\color{gray!10},
    showspaces=false,
    showstringspaces=false,
    showtabs=false,
    frame=single,
    tabsize=4,
    captionpos=b,
    breaklines=true,
    breakatwhitespace=false,
    escapeinside={\%*}{*)}
}


\begin{document}


\begin{titlepage}
    \centering
    
    \vspace*{1cm}
    {\fontsize{20}{24}\bfseries University of Dhaka}\\[0.4cm]
    {\large Department of Computer Science and Engineering}\\[1cm]
    
    \hline
    \vspace{.5cm}
    {\Large \textbf{CSE 3113: Microprocessor and Assembly Lab}\\[.5cm]}
    \Large{Report of Tasks from Lab 8}
    
    \vspace{.5cm}
    \hline
    
    \vspace{1.5cm}
    
    % {\large \textbf{Lab Group: 03}}\\[0.5cm]
    
    \begin{center}
        \textbf{Submitted By} \\
        Muhaiminul Islam Ninad \\
        Roll:  43
    \end{center}
    
    
    \vspace{1.5cm}
    
    {\large \textbf{Submitted To:}}\\[0.4cm]
    Dr. Upama Kabir \\
    Professor,  Dept. of CSE, University of Dhaka\\[.3cm]
    Dr. Mosarrat Jahan \\
    Associate Professor, Dept. of CSE, University of Dhaka \\[.3cm]
    Mr. Jargis Ahmed, \\
    Lecturer, Dept. of CSE, University of Dhaka\\[.3cm]
    Mr. Palash Roy, \\
    Lecturer, Dept. of CSE, University of Dhaka\\[1.3cm]
    
    {\large \textbf{Submission Date:} \today}
    
    \vfill
    
    \thispagestyle{empty}
\end{titlepage}


\newpage
\tableofcontents
\newpage

\section{Problem 1}
Write an assembly language to create a 2D Array. Also use the 2D Array translation
formula to access the Array elements making use of register indirect addressing mode.

\subsection{Screenshot at the start of execution}

\includegraphics[width=\textwidth]{images/1.1.png}

\subsection{Screenshot at the end of execution}

\includegraphics[width=\textwidth]{images/1.2.png}

\newpage
\section{Problem 2}

Write an assembly language to perform the multiplication of two matrices.

\subsection{Screenshot at the start of execution}
\includegraphics[width=\textwidth]{images/2.1.png}

\subsection{Screenshot at the end of execution}
\includegraphics[width=\textwidth]{images/2.2.png}

\newpage
\section{Problem 3}

Write an assembly language in which four bytes of data are stored in memory location.
Add all data bytes and use register r5 to store any carry generated while adding data
bytes by calling a function Add_byte

\subsection{Screenshot at the start of execution}
\includegraphics[width=\textwidth]{images/3.1.png}
\subsection{Screenshot at the end of execution}
\includegraphics[width=\textwidth]{images/3.2.png}

\newpage
\section{Problem 4}

Write an assembly language which convert BCD data to Binary data by calling a
function BCD_binary.

\subsection{Screenshot at the start of execution}

\includegraphics[width=\textwidth]{images/4.1.png}
\subsection{Screenshot at the end of execution}
\includegraphics[width=\textwidth]{images/4.2.png}


\newpage
\section{Problem 5}

Write an assembly language to implement a counter to count from ’00 – 99’ (UP-
COUNTER) in BCD and also to generate a delay of one second between the counts.

\subsection{Screenshot at the start of execution}
\includegraphics[width=\textwidth]{images/5.1.png}
\subsection{Screenshot at the end of execution}
\includegraphics[width=\textwidth]{images/5.2.png}



\end{document}